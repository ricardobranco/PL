\documentclass[11pt, a4paper]{report}
\usepackage{graphicx, url,hyperref,verbatim}
\usepackage[utf8]{inputenc}
\usepackage[portuges]{babel}
\setlength{\parskip}{1ex}
\setlength{\parindent}{0ex}
\setcounter{secnumdepth}{6}
\setcounter{tocdepth}{6}
\makeatletter
\newcounter{subsubparagraph}[subparagraph]
\renewcommand\thesubsubparagraph{%%
\thesubparagraph.\@arabic\c@subsubparagraph}
\newcommand\subsubparagraph{%%
\@startsection{subsubparagraph}    %% counter
{6} %% level
{\parindent}                     %% indent
{3.25ex \@plus 1ex \@minus .2ex} %% beforeskip
{-1em}                           %% afterskip
{\normalfont\normalsize\bfseries}}
\newcommand\l@subsubparagraph{\@dottedtocline{6}{10em}{5em}}
\newcommand{\subsubparagraphmark}[1]{}
\makeatother
\title{Relatório - Processamento de Linguagens\\ 
 Report 2007: vamos escrever relatórios}
\author{
Diogo Alves
\\ \texttt{A61030}
\\ \texttt{a61030@alunos.uminho.pt}
\and 
Helder Gonçalves
\\ \texttt{A61084}
\\ \texttt{a61084@alunos.uminho.pt}
\and 
Ricardo Branco
\\ \texttt{A61075}
\\ \texttt{a61075@alunos.uminho.pt}
}
\date{\today}
\begin{document}
\maketitle
\begin{abstract}Neste trabalho tem-se como objetivo criar um analisador léxico e um sintático, que "processa"/analisa o texto, que apanha as palavras reservadas, e de seguida verifica se a estrutura do relatório está bem construida. Enquanto analisa o texto este é guardado em estruturas de dados, e em listas ligadas para separar o código html do código latex, convertendo e criando por fim ficheiros HTML e/ou LaTeX com o nosso relatório convertido para cada uma das linguagens.\\
\end{abstract}\tableofcontents
\listoffigures
\listoftables
\chapter{Introdução}
Para o segundo Trabalho Prático da Unidade Curricular de Processamento de Linguagens, a nossa escolha foi o enunciado 3 que tem como titulo: "Report 2007: vamos escrever relatórios".\\
Neste projeto, pretende-se que seja criado um compilador capaz de "converter" uma relatório escrito numa linguagem criada por nós, e já usada no trabalho prático 1 para a linguagem HTML ou/e LaTeX.\\
Portanto, neste documento irão estar presentes as nossas decisões, a estruturação do projecto, bem como as explicações e funcionamento do mesmo.\\
\chapter{Sintaxe da Nossa Linguagem}
Nós, como referido anteriormente, estamos a "continuar" o trabalho realizado no Trabalho Prático 1 (TP1) e por isso a sintaxe da linguagem manteve-se a do trabalho anterior, fazendo algumas alterações e acrescentando outras funcionalidades que são as seguintes:\\
\begin{figure}[!hbp]
\includegraphics{sintaxe.png\caption{1ª parte}
\end{figure}
\end{document}
